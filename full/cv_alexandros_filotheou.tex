\documentclass[a4paper,10pt,twoside]{article}
\usepackage{alefilot}
\usepackage{fullpage}
\usepackage{fontspec}
\usepackage[utf8]{inputenc}

%\defaultfontfeatures{Mapping=tex-text}
%\setmainfont[Mapping=tex-text]{GFS Didot} % Main document font
\setmainfont{Latin Modern Roman} % Main document font

%\usepackage[big]{layaureo} % Margin formatting of the A4 page, an alternative to layaureo can be
\usepackage{fullpage}
% To reduce the height of the top margin uncomment:
\addtolength{\voffset}{-1.3cm}

\usepackage{hyperref} % Required for adding links	and customizing them
\definecolor{linkcolour}{rgb}{0,0.2,0.6} % Link color
\hypersetup{colorlinks,breaklinks,urlcolor=linkcolour,linkcolor=linkcolour} % Set link colors throughout the document

\usepackage{titlesec} % Used to customize the \section command
\titleformat{\section}{\Large\scshape\raggedright}{}{0em}{}[\titlerule] % Text formatting of sections
\titlespacing{\section}{0pt}{3pt}{3pt} % Spacing around sections


\usepackage{fancyhdr}
\setlength{\headheight}{15.2pt}
\setlength{\textheight}{710pt}
\pagestyle{fancy}
\fancyhead[]{}
\renewcommand{\headrulewidth}{0pt}

% FOOTER
\usepackage{fancyhdr}
\setlength{\headheight}{15.2pt}
\setlength{\textheight}{710pt}
\setlength{\footskip}{60pt} % Distance from bottom of text to footer
\pagestyle{fancy}
\fancyhead[]{}
\renewcommand{\headrulewidth}{0pt}
\fancyfoot[LO,RE]{\footnotesize Thessaloniki, Greece. AUG 2025  \hfill contact: (+30) 693 87 87 677 $|$ alexandros.filotheou@gmail.com\\
 \hfill source: \href{https://github.com/li9i/cv}{https://github.com/li9i/cv}}


\usepackage{tocloft}
\setcounter{tocdepth}{1}

% Custom ToC appearance
\setlength{\cftbeforesecskip}{4pt}
\setlength{\cftbeforesubsecskip}{2pt}
\setlength{\cftsecindent}{0pt}
\setlength{\cftsubsecindent}{20pt}
\setlength{\cftsecnumwidth}{3em}
\setlength{\cftsubsecnumwidth}{4em}
\renewcommand{\cftsecfont}{\bfseries}
\renewcommand{\cftsubsecfont}{\itshape}
\renewcommand{\cftsecpagefont}{\bfseries}


\usepackage{wasysym}
\usepackage{fontawesome}
\usepackage{tikz}

\usepackage[backend=biber,style=alphabetic,sorting=ynt,babel=other,bibencoding=utf8, language=autobib, maxnames=99]{biblatex}
\addbibresource{papers_during_phd.bib}
\begin{filecontents}{papers_during_phd.bib}
@inproceedings{Filotheou2018,
author = {Filotheou, Alexandros and Nikou, Alexandros and Dimarogonas, Dimos V.},
booktitle = {2018 European Control Conference (ECC)},
doi = {10.23919/ECC.2018.8550343},
isbn = {978-3-9524-2698-2},
month = {jun},
pages = {8--13},
publisher = {IEEE},
title = {{Decentralized Control of Uncertain Multi-Agent Systems with Connectivity Maintenance and Collision Avoidance}},
url = {https://ieeexplore.ieee.org/document/8550343/},
year = {2018}
}
@article{Filotheou2020,
author = {Filotheou, Alexandros and Nikou, Alexandros and Dimarogonas, Dimos V.},
doi = {10.1080/00207179.2018.1514129},
issn = {0020-7179},
journal = {International Journal of Control},
month = {jun},
number = {6},
pages = {1470--1484},
title = {{Robust decentralised navigation of multi-agent systems with collision avoidance and connectivity maintenance using model predictive controllers}},
url = {https://www.tandfonline.com/doi/full/10.1080/00207179.2018.1514129},
volume = {93},
year = {2020}
}
@article{Tzitzis2020,
author = {Tzitzis, Anastasios and Megalou, Spyros and Siachalou, Stavroula and Emmanouil, Tsardoulias G. and Filotheou, Alexandros and Yioultsis, Traianos V. and Dimitriou, Antonis G.},
doi = {10.1109/JRFID.2020.3000332},
issn = {2469-7281},
journal = {IEEE Journal of Radio Frequency Identification},
month = {dec},
number = {4},
pages = {283--299},
title = {{Trajectory Planning of a Moving Robot Empowers 3D Localization of RFID Tags With a Single Antenna}},
url = {https://ieeexplore.ieee.org/document/9109328/},
volume = {4},
year = {2020}
}
@inproceedings{Tzitzis2020a,
author = {Tzitzis, Anastasios and Filotheou, Alexandros and Siachalou, Stavroula and Tsardoulias, Emmanouil and Megalou, Spyros and Bletsas, Aggelos and Panayiotou, Konstantinos and Symeonidis, Andreas and Yioultsis, Traianos and Dimitriou, Antonis G.},
booktitle = {2020 IEEE International Conference on RFID (RFID)},
doi = {10.1109/RFID49298.2020.9244904},
isbn = {978-1-7281-5576-0},
month = {sep},
pages = {1--8},
publisher = {IEEE},
title = {{Real-time 3D localization of RFID-tagged products by ground robots and drones with commercial off-the-shelf RFID equipment: Challenges and Solutions}},
url = {https://ieeexplore.ieee.org/document/9244904/},
year = {2020}
}
@inproceedings{Mylonopoulos2021,
author = {Mylonopoulos, George and Chatzistefanou, Aristidis Raptopoulos and Filotheou, Alexandros and Tzitzis, Anastasios and Siachalou, Stavroula and Dimitriou, Antonis G.},
booktitle = {2021 IEEE International Conference on RFID Technology and Applications (RFID-TA)},
doi = {10.1109/RFID-TA53372.2021.9617436},
isbn = {978-1-6654-2657-2},
month = {oct},
pages = {32--35},
publisher = {IEEE},
title = {{Localization, Tracking and Following a Moving Target by an RFID Equipped Robot}},
url = {https://ieeexplore.ieee.org/document/9617436/},
year = {2021}
}
@inproceedings{Dimitriou2021a,
author = {Dimitriou, Antonis and Tzitzis, Anastasios and Filotheou, Alexandros and Megalou, Spyros and Siachalou, Stavroula and Chatzistefanou, Aristidis R. and Malama, Andreana and Tsardoulias, Emmanouil and Panayiotou, Konstantinos and Giannelos, Evaggelos and Vasiliadis, Thodoris and Mouroutsos, Ioannis and Karanikas, Ioannis and Petrou, Loukas and Symeonidis, Andreas and Sahalos, John and Yioultsis, Traianos and Bletsas, Aggelos},
booktitle = {2021 6th International Conference on Smart and Sustainable Technologies (SpliTech)},
doi = {10.23919/SpliTech52315.2021.9566425},
isbn = {978-953-290-112-2},
month = {sep},
pages = {01--06},
publisher = {IEEE},
title = {{Autonomous Robots, Drones and Repeaters for Fast, Reliable, Low-Cost RFID Inventorying {\&} Localization}},
url = {https://ieeexplore.ieee.org/document/9566425/},
year = {2021}
}
@article{Tzitzis2023,
author = {Tzitzis, Anastasios and Filotheou, Alexandros and Chatzistefanou, Aristidis Raptopoulos and Yioultsis, Traianos and Dimitriou, Antonis G.},
doi = {10.1109/JRFID.2023.3288982},
issn = {2469-7281},
journal = {IEEE Journal of Radio Frequency Identification},
pages = {1--1},
title = {{Real-Time Global Localization of a Mobile Robot by Exploiting RFID Technology}},
url = {https://ieeexplore.ieee.org/document/10160120/},
year = {2023}
}
@INPROCEEDINGS{8739423,
  author={Tzitzis, Anastasios and Megalou, Spyros and Siachalou, Stavroula and Yioultsis, Traianos and Kehagias, Athanasios and Tsardoulias, Emmanouil and Filotheou, Alexandros and Symeonidis, Andreas and Petrou, Loukas and Dimitriou, Antonis G.},
  booktitle={2019 13th European Conference on Antennas and Propagation (EuCAP)},
  title={Phase ReLock - Localization of RFID Tags by a Moving Robot},
  year={2019},
  volume={},
  number={},
  pages={1-5},
  doi={}}
@INPROCEEDINGS{8739486,
  author={Megalou, Spyros and Tzitzis, Anastasios and Siachalou, Stavroula and Yioultsis, Traianos and Sahalos, John and Tsardoulias, Emmanouil and Filotheou, Alexandros and Symeonidis, Andreas and Petrou, Loukas and Bletsas, Aggelos and Dimitriou, Antonis G.},
  booktitle={2019 13th European Conference on Antennas and Propagation (EuCAP)},
  title={Fingerprinting Localization of RFID tags with Real-Time Performance-Assessment, using a Moving Robot},
  year={2019},
  volume={},
  number={},
  pages={1-5},
  doi={}}
@article{Filotheou2020bA,
author = {Filotheou, Alexandros and Tsardoulias, Emmanouil and Dimitriou, Antonis and Symeonidis, Andreas and Petrou, Loukas},
doi = {10.1007/s10846-019-01086-y},
issn = {0921-0296},
journal = {Journal of Intelligent {\&} Robotic Systems},
month = {jun},
number = {3-4},
pages = {567--601},
title = {{Quantitative and Qualitative Evaluation of ROS-Enabled Local and Global Planners in 2D Static Environments}},
url = {http://link.springer.com/10.1007/s10846-019-01086-y},
volume = {98},
year = {2020}
}
@article{Filotheou2020cA,
author = {Filotheou, Alexandros and Tsardoulias, Emmanouil and Dimitriou, Antonis and Symeonidis, Andreas and Petrou, Loukas},
doi = {10.1007/s10846-020-01253-6},
issn = {0921-0296},
journal = {Journal of Intelligent {\&} Robotic Systems},
month = {dec},
number = {3-4},
pages = {925--944},
title = {{Pose Selection and Feedback Methods in Tandem Combinations of Particle Filters with Scan-Matching for 2D Mobile Robot Localisation}},
url = {https://link.springer.com/10.1007/s10846-020-01253-6},
volume = {100},
year = {2020}
}
@article{Filotheou2022eA,
author = {Filotheou, Alexandros and Tzitzis, Anastasios and Tsardoulias, Emmanouil and Dimitriou, Antonis and Symeonidis, Andreas and Sergiadis, George and Petrou, Loukas},
doi = {10.1007/s10846-021-01535-7},
issn = {0921-0296},
journal = {Journal of Intelligent {\&} Robotic Systems},
month = {feb},
number = {2},
pages = {26},
title = {{Passive Global Localisation of Mobile Robot via 2D Fourier-Mellin Invariant Matching}},
url = {https://link.springer.com/10.1007/s10846-021-01535-7},
volume = {104},
year = {2022}
}
@article{Filotheou2022,
author = {Filotheou, Alexandros},
doi = {10.1016/j.robot.2021.103957},
issn = {09218890},
journal = {Robotics and Autonomous Systems},
month = {mar},
pages = {103957},
title = {{Correspondenceless scan-to-map-scan matching of homoriented 2D scans for mobile robot localisation}},
url = {https://linkinghub.elsevier.com/retrieve/pii/S0921889021002323},
volume = {149},
year = {2022}
}
@inproceedings{Filotheou2022iA,
author={Filotheou, Alexandros and Sergiadis, Georgios D. and Dimitriou, Antonis G.},
booktitle={2022 IEEE/RSJ International Conference on Intelligent Robots and Systems (IROS)},
title={FSM: Correspondenceless scan-matching of panoramic 2D range scans},
month={oct},
year={2022},
pages={6968-6975},
doi={10.1109/IROS47612.2022.9981228}
}
@INPROCEEDINGS{Filotheou2024,
  author={Filotheou, Alexandros},
  booktitle={2024 IEEE/RSJ International Conference on Intelligent Robots and Systems (IROS)},
  title={CBGL: Fast Monte Carlo Passive Global Localisation of 2D LIDAR Sensor},
  year={2024},
  pages={3268-3275},
  doi={10.1109/IROS58592.2024.10802235}
}
@article{Filotheou2023A,
author = {Filotheou, Alexandros and Symeonidis, Andreas L. and Sergiadis, Georgios D. and Dimitriou, Antonis G.},
doi = {10.1016/j.array.2023.100288},
issn = {25900056},
journal = {Array},
month = {jul},
pages = {100288},
title = {{Correspondenceless scan-to-map-scan matching of 2D panoramic range scans}},
url = {https://linkinghub.elsevier.com/retrieve/pii/S2590005623000139},
volume = {18},
year = {2023}
}

\end{filecontents}
\usepackage{filecontents}
\usepackage{bibentry}


\DeclareBibliographyCategory{nobibliography}
\addtocategory{nobibliography}{Tzitzis2020}
\addtocategory{nobibliography}{Tzitzis2020a}
\addtocategory{nobibliography}{Mylonopoulos2021}
\addtocategory{nobibliography}{Dimitriou2021a}
\addtocategory{nobibliography}{Tzitzis2023}
\addtocategory{nobibliography}{8739423}
\addtocategory{nobibliography}{8739486}
\addtocategory{nobibliography}{Filotheou2020bA}
\addtocategory{nobibliography}{Filotheou2020cA}
\addtocategory{nobibliography}{Filotheou2022eA}
\addtocategory{nobibliography}{Filotheou2022iA}
\addtocategory{nobibliography}{Filotheou2023A}
\addtocategory{nobibliography}{Filotheou2024}
\addbibresource{papers_during_phd.bib}


% Make author bold
\def\makenamesetup{%
  \def\bibnamedelima{~}%
  \def\bibnamedelimb{ }%
  \def\bibnamedelimc{ }%
  \def\bibnamedelimd{ }%
  \def\bibnamedelimi{ }%
  \def\bibinitperiod{.}%
  \def\bibinitdelim{~}%
  \def\bibinithyphendelim{.-}}
\newcommand*{\makename}[2]{\begingroup\makenamesetup\xdef#1{#2}\endgroup}

\newcommand*{\boldname}[3]{%
  \def\lastname{#1}%
  \def\firstname{#2}%
  \def\firstinit{#3}}
\boldname{}{}{}

% Patch new definitions
\renewcommand{\mkbibnamegiven}[1]{%
  \ifboolexpr{ ( test {\ifdefequal{\firstname}{\namepartgiven}} or test {\ifdefequal{\firstinit}{\namepartgiven}} ) and test {\ifdefequal{\lastname}{\namepartfamily}} }
  {\mkbibbold{#1}}{#1}%
}

\renewcommand{\mkbibnamefamily}[1]{%
  \ifboolexpr{ ( test {\ifdefequal{\firstname}{\namepartgiven}} or test {\ifdefequal{\firstinit}{\namepartgiven}} ) and test {\ifdefequal{\lastname}{\namepartfamily}} }
  {\mkbibbold{#1}}{#1}%
}

\boldname{Filotheou}{Alexandros}{A.}





























\begin{document}

%\pagestyle{plain} % Removes page numbering
\pagenumbering{gobble}

%----------------------------------------------------------------------------------------
%	NAME AND CONTACT INFORMATION
%----------------------------------------------------------------------------------------

\par{\centering{\Huge Dr Alexandros Philotheou}\bigskip\par}
\vspace{4em}
\begin{center}
\begin{tabular}{rp{10cm}}
Current location and age                & Thessaloniki, Greece | $38$ \\
Phone                                   & (+30) 693 87 87 677 \\
e-mail                                  & \href{mailto:alexandros.filotheou@gmail.com}{alexandros.filotheou@gmail.com}
\end{tabular}
\end{center}

\vfill

%----------------------------------------------------------------------------------------
%	SUMMARY
%----------------------------------------------------------------------------------------
\vspace{-0.5cm}
\begin{bw_box} \small
  \hspace{1em}I have 7+ years of hands-on experience in robotics which includes SLAM, \href{https://github.com/li9i/fsm-lo}{Localisation},
  \href{https://link.springer.com/article/10.1007/s10846-019-01086-y}{Autonomous Navigation}, \href{https://www.tandfonline.com/doi/full/10.1080/00207179.2018.1514129}{Control},
  \href{https://github.com/li9i/pandora\_vision\_2014}{Computer Vision}, and
  general integration, problem-solving, and troubleshooting. These I have acquired in \href{https://relief.web.auth.gr/\%CF\%81\%CE\%BF\%CE\%BC\%CF\%80\%CE\%BF\%CF\%84\%CE\%B9\%CE\%BA\%CE\%AC-\%CE\%BF\%CF\%87\%CE\%AE\%CE\%BC\%CE\%B1\%CF\%84\%CE\%B1/}{real
  conditions with real robots} as well as in simulation, working for research
  projects funded by the European Commission and the Greek State, or through volunteering. I am proficient in ROS and ROS 2, but also MATLAB/Octave, all the while under Linux, which has been my primary OS since 2008. My
  primary coding language is \href{https://github.com/li9i/fsm}{C++} and my
  secondary
  \href{https://github.com/cultureid-auth-ros-packages/cultureid-waypoints-following}{Python}.

  \hspace{1em}I hold a \href{https://ikee.lib.auth.gr/record/291560}{diploma} and a \href{https://ikee.lib.auth.gr/record/354644}{PhD} in Electrical and Computer Engineering and I
  am qualified with a
  \href{http://kth.diva-portal.org/smash/record.jsf?pid=diva2\%3A1102597\&dswid=2875}{Master's degree in Control and Robotics}.
  I have deliberately chosen the academic $+$ software engineering path because I
  wanted to reconcile the two dominant types of engineers: the academic type
  whose coding boundary is MATLAB, and the software engineering type whose
  theoretic boundary remains in code.

  \hspace{1em}It is my pleasure to document my motivation and contributions in a clear and concise
  manner that acknowledges my audience's existing knowledge with regard to
  \href{https://github.com/li9i/pandora_vision_2014/blob/hydro-devel/pandora_vision_hole_detector/src/hole_fusion_node/hole_fusion.cpp}{code docmentation},
  \href{https://www.youtube.com/watch?v=xaDKjI0WkDc}{presentations}, \href{https://ieeexplore.ieee.org/abstract/document/9981228}{technical papers},
  or \href{https://github.com/li9i/ros1_humble_bridge_template}{tutorials}. Most of all I enjoy being a member of a team and solving problems together with other people.
\end{bw_box}

\vfill

%----------------------------------------------------------------------------------------
%	LINKS
%----------------------------------------------------------------------------------------
\section*{Quick Links}

\begin{tabular}{rp{12cm}}

Indicative software packages: &
\href{https://github.com/li9i/cbgl}{\texttt{cbgl}} $\cdot$
\href{https://github.com/li9i/fsm-lo}{\texttt{fsm-lo}} $\cdot$
\href{https://github.com/li9i/lama-odom}{\texttt{lama-odom}} $\cdot$
\href{https://github.com/li9i/pandora\_vision\_2014/tree/hydro-devel/pandora\_vision\_hole\_detector}{\texttt{pvhd}} ---  \href{https://github.com/li9i}{\texttt{github}} \\


Demos / videos: &
\href{https://www.youtube.com/watch?v=xaDKjI0WkDc}{\texttt{cbgl}} $\cdot$ \href{https://cultureid.web.auth.gr/?page\_id=200&lang=en}{\texttt{cultureid}} $\cdot$ \href{https://www.youtube.com/watch?v=hB4qsHCEXGI}{\texttt{fsm}} $\cdot$ \href{https://relief.web.auth.gr/}{\texttt{relief}} $\cdot$ \href{https://www.youtube.com/watch?v=937OZez1iN8}{\texttt{mpc}}\\

Indicative publications: &
\href{https://ieeexplore.ieee.org/abstract/document/9981228}{[1]}
\href{https://www.sciencedirect.com/science/article/abs/pii/S0921889021002323}{[2]}
\href{https://www.tandfonline.com/doi/full/10.1080/00207179.2018.1514129}{[3]} --- \href{https://scholar.google.com/citations?view\_op=list\_works\&hl=en\&user=9\_hI4hMAAAAJ}{google scholar}\\

  Portfolio & \href{https://docs.google.com/viewer?url=https://raw.githubusercontent.com/li9i/portfolio/master/portfolio.pdf}{On \texttt{github}}
\end{tabular}
\\

\vfill


\tableofcontents

\newpage

%----------------------------------------------------------------------------------------
%	WORK EXPERIENCE
%----------------------------------------------------------------------------------------
\vspace*{\fill}

\section{Work Experience}


\begin{tabular}{rp{11cm}}
$2023.09 - \hfill \textnormal{present} \hfill$ & \textbf{Robotics Engineer} \\
                                                           & Center for Research and Technology Hellas (CERTH), Thessaloniki, Greece\\
&\\
$2018.09 - 2023.03$ & \textbf{Robotics and Control Engineer} \\
                    & Aristotle University of Thessaloniki, Greece\\
                    & Electrical and Computer Engineering Department \\
&\\
$2016.09 - 2016.11$ & \textbf{Teaching Assistant} $\cdot$  DD2380 Artificial Intelligence \\
                    & KTH Royal Institute of Technology, Stockholm, Sweden\\
&\\
%------------------------------------------------

$2011.10 - 2012.03$ & \textbf{Database Designer} \\
                    & Egnatia Motorway S.A., Thessaloniki, Greece \\
& \small{Design and implementation of a unified database, suitable for the needs of the
Instrumental Landslide and Geotechnical Issues
monitoring system, in the context of the European Research Program IRIS.
}\\
\multicolumn{2}{c}{} \\

%------------------------------------------------

$2011.03 - 2011.05$ & \textbf{Database Developer} \\
                    & Internship $\cdot$ Egnatia Motorway S.A., Thessaloniki, Greece\\
& \small{Design, development and technical and user documentation of a system for data
recovery and report-issuing from the company's bridge register using
customizable criteria. The application was developed using ORACLE developer tools.
}\\
\multicolumn{2}{c}{} \\

%------------------------------------------------

$2008.07 - 2009.06$ & \textbf{Telecommunications Engineer} \\
  & Hellenic Telecommunications Organisation (OTE S.A.) Thessaloniki, Greece\\
& \small{Remote service in matters of local and wide area networks.
}\\
&\\
\end{tabular}

\vspace*{\fill}
\newpage

%----------------------------------------------------------------------------------------
%	PROJECTS
%----------------------------------------------------------------------------------------

\vspace*{\fill}
\section{Experience in R\&D Projects}

\begin{tabular}{rp{11cm}}
$2023.09 - \text{present}$ & \href{https://www.robetarme-project.eu/}{\textbf{RoBétArmé}} \\
                           & \textit{Center for Research and Technology Hellas (CERTH), Thessaloniki, Greece}\\
\\
                           & Head of SW integration. Maintainer of project's GitLab repositories. Development of ROS packages for planning, scheduling, orchestrating and triggering ROS Noetic and ROS 2 Humble packages and other software, and their integation with Behaviour Trees. Integration onto three mobile platforms (two ROBOTNIK mobile bases and one trailer). Integration of in-house and third-party ROS Noetic and ROS 2 Humble packages. Work with 3D LIDARs, RGBD cameras, robot URDFs. Translation of custom interfaces between ROS 1 and ROS 2. Network configuration and troubleshooting for ROS 2 via DDS, Zenoh. Virtualisation via Docker and template specification for all project images. General software and hardware troubleshooting. Documentation. \\
&\\
$2020.04 - 2023.03$ & \href{https://cultureid.web.auth.gr/?page\_id=216\&lang=en}{\textbf{CULTUREID}} \\
                    & \textit{Aristotle University of Thessaloniki, Greece}\\
                    & \textit{Electrical and Computer Engineering Department} \\
\\
                    & Development of ROS packages for mapping, localisation, navigation, visualisation (Tkinter). Integration onto one mobile platform (Turtlebot). Sole robot-related integrator. Integration of RFID-related factory- and custom-built software and wrapping into ROS. General software and hardware troubleshooting and integration. Work with 2D LIDARs, RGBD cameras, RFID readers. Documentation. \\
&\\
$2018.09 - 2021.08$ & \href{https://relief.web.auth.gr/}{\textbf{RELIEF}} \\
                    & \textit{Aristotle University of Thessaloniki, Greece}\\
                    & \textit{Electrical and Computer Engineering Department} \\
\\
                    & Development of ROS packages for mapping, localisation, navigation, and visualisation (Qt). Integration onto two mobile platforms (Turtlebot \& Robotnik RB1). Sole robot-related integrator. Integration of RFID-related factory- and custom-built software and wrapping into ROS. General software and hardware troubleshooting and integration. Work with 2D LIDARs, RGBD cameras, RFID readers. Documentation. \\
&\\
\end{tabular}

%----------------------------------------------------------------------------------------
%	VOLUNTARY WORK
%----------------------------------------------------------------------------------------

\section{Voluntary Experience}


\begin{tabular}{rp{11cm}}
  $2013.10 -2014.07$ & \textbf{Computer Vision Engineer} $\cdot$ PANDORA Robotics Undergrad Team,
Electrical and Computer Engineering Department, Aristotle University of Thessaloniki\\
& \small{Design of the architecture, implementation and thorough documentation of
the Hole Detection system of the PANDORA robot under ROS, using RGB+Depth sensors
(Microsoft Kinect and ASUS Xtion) in the context of the conditions of the international
RoboCup Rescue competition.}\\
\multicolumn{2}{c}{} \\
\end{tabular}

\vspace*{\fill}
\newpage

%----------------------------------------------------------------------------------------
%	EDUCATION
%----------------------------------------------------------------------------------------

\vspace*{\fill}
\section{Education}

\begin{tabular}{rp{11cm}}
$2018.09 - 2023.06$  & \textbf{Doctorate} \\
                     & Aristotle University of Thessaloniki, Greece \\
                     & Electrical and Computer Engineering Department \\
                     & \\
                     & \begin{small}\textbf{Thesis} $\cdot$ 2D LIDAR sensor pose estimation via scan--to--map-scan matching \end{small} \\
                     & \begin{small}Advisor: Prof. Georgios Sergiadis, Department of Telecommunications\end{small} \\
                     & \begin{small}Defended: 28/06/2023\end{small}\\
                     & \begin{small}Commitee: Georgios Sergiadis (AUTh),
                     Andreas Symeonidis (AUTh), Traianos Yioultsis (AUTh),
                     Zoe Doulgeri (AUTh), Nikolaos Fachantidis (UoM), Aggelos Bletsas (TUC), Anastasios Delopoulos (AUTh) \end{small}\\
&\\
$2015.09 - 2017.06$  & \textbf{Master of Science} \\
                     & KTH Royal Institute of Technology, Stockholm, Sweden\\
                     & School of Electrical Engineering\\
                     & Programme title: \textit{Systems, Control, and Robotics} \\
                     & \\
                     & \begin{small}\textbf{Thesis} $\cdot$ Robust Decentralized Control of Cooperative Multi-robot Systems:
                       An inter-constraint Receding Horizon approach\end{small} \\
                     & \begin{small}Advisor: Prof. Dimos Dimarogonas, Department of Automatic Control\end{small} \\
&\\
$2005.09 - 2013.07$  & \textbf{Diploma} \\
                     & Aristotle University of Thessaloniki, Greece \\
                     & Electrical and Computer Engineering Department \\
                     & GPA: $7.94 / 10.0$ \\
                     & Class rank: $23 / 280$ $-$ $92^{nd}$ percentile \\
                     &\\
                     & \begin{small}\textbf{Thesis} $\cdot$ Multi-label classification using Learning Classifier Systems \end{small}\\
                     & \begin{small}Advisor: Prof. Pericles Mitkas, Department of Electronics and Computer Engineering\end{small}\\
                     & \begin{small}Commitee: Pericles Mitkas (AUTh), Anastasios Delopoulos (AUTh), Andreas Symeonidis (AUTh) \end{small}\\
&\\
%$09.1999 - 06.2005$ & \textbf{Secondary education diploma} \\
                    %& Anatolia High School, Thessaloniki \\
                    %& GPA: $19.0 / 20.0$ \\
                    %&\\

%------------------------------------------------

\end{tabular}

\vspace*{\fill}
\newpage


%----------------------------------------------------------------------------------------
%	Publications
%----------------------------------------------------------------------------------------
\section{Publications}

\href{https://scholar.google.com/citations?view\_op=list\_works\&hl=en\&user=9\_hI4hMAAAAJ}{Link to Google Scholar}\\

{
\textcolor{magenta}{
  \fullcite{Filotheou2024}}\\

\fullcite{Tzitzis2023}\\

\fullcite{Filotheou2023A}\\

\textcolor{magenta}{
  \fullcite{Filotheou2022iA}}\\

\textcolor{magenta}{
  \fullcite{Filotheou2022}}\\

\fullcite{Filotheou2022eA}\\

\fullcite{Mylonopoulos2021}\\

\fullcite{Dimitriou2021a}\\

\fullcite{Filotheou2020cA}\\

\fullcite{Tzitzis2020}\\

\fullcite{Tzitzis2020a}\\

\fullcite{Filotheou2020bA}\\

\fullcite{8739423}\\

\fullcite{8739486}\\

\textcolor{magenta}{
\fullcite{Filotheou2020} }\\

\fullcite{Filotheou2018}
}\\


\vspace*{\fill}
\newpage

\vspace*{\fill}
%----------------------------------------------------------------------------------------
%	Distinctions
%----------------------------------------------------------------------------------------

\section{Distinctions}
\begin{tabular}{rl}

$2024$ & One of eighteen authors of single-authored papers (1,587 total) presented at the \\ &
         2024 IEEE/RSJ International Conference on Intelligent Robots and Systems (IROS 2024) \\

%------------------------------------------------

$2016$ & Teaching Assistant, DD2380 - Artificial Intelligence, \\ & under the
supervision of Professor Patric Jensfelt, KTH Royal Institute of Technology, Sweden \\

%------------------------------------------------

$2015$ & $2^{nd}$ place in Autonomy class in RoboCup Rescue as member of PANDORA robotics team \\

%------------------------------------------------

$2013$ & Ranked $30^{th}$ in graduating class among $224$ students who graduated in $2013$, ECE, AUTh, Greece \\

%------------------------------------------------

$2011$ & Top of my class in the course of Database Systems, winter semester $2010 - 2011$,
AUTh, Greece \\

%------------------------------------------------

$2005$ & Ranked $21^{st}$ in entering class among $280$ students who enrolled in $2005$, ECE, AUTh, Greece\\
&\\
\end{tabular}



%----------------------------------------------------------------------------------------
%	Seminars
%----------------------------------------------------------------------------------------

%\section{Seminars \& Conferences Attendance}

%\begin{tabular}{rp{12cm}}
%$06.2012$ & Seminar in Bioinformatics -
%Department of Electrical and Computer Engineering, Aristotle University of Thessaloniki, Thessaloniki, Greece \\

%$10.2011 - 01.2012$ & Seminars for developing web and mobile applications,
%Independent student initiative at the School of Electrical \& Computer Engineering, Aristotle
%University of Thessaloniki, Thessaloniki, Greece \\

%$04.2009$ & $3^{rd}$ Panhellenic Electrical \& Computer Engineering Students Conference,
%Aristotle University of Thessaloniki, Thessalonki, Greece \\

%$04.2008$ & $2^{nd}$ Panhellenic Electrical \& Computer Engineering Students Conference,
%National Technical University of Athens (NTUA), Athens, Greece \\

%&\\
%\end{tabular}




%----------------------------------------------------------------------------------------
%	COMPUTER SKILLS
%----------------------------------------------------------------------------------------
\section{Computer Skills}

\begin{tabular}{rp{9cm}}
  Languages & \texttt{C/C++}, \texttt{Python}, \texttt{shell}, \texttt{MATLAB/Octave}, $\{$\texttt{PL/}$\}$\texttt{SQL}, \texttt{Java}, \texttt{Assembly}\\
&\\
  $\{$Meta-$\}$operating Systems & \texttt{Linux}, \texttt{ROS} \\
&\\
  Graphics & \texttt{AutoCAD}, \texttt{Gimp}\\
&\\
  Tools & \texttt{git}, \texttt{Docker}, \texttt{OpenCV}, \texttt{Qt (cpp)}, \texttt{Tkinter (py)}, Behavior Trees, \LaTeX, \texttt{Oracle Forms} / \texttt{Reports}, \texttt{Microsoft} $\{$\texttt{Visio}, \texttt{Project}, \texttt{Office}$\}$ \\
&\\
  Control & MPC, LQR, PID
\end{tabular}\\


%----------------------------------------------------------------------------------------
%	LANGUAGES
%----------------------------------------------------------------------------------------

\section{Languages}

\begin{tabular}{rp{12cm}}
English & Fluent - IELTS Score 8.5\\
Greek & Mother tongue
\end{tabular}\\

%%----------------------------------------------------------------------------------------
%%	LINKS
%%----------------------------------------------------------------------------------------
%\section{Links}

%\begin{tabular}{rp{12cm}}
%Indicative publications: &
%\href{https://ieeexplore.ieee.org/abstract/document/9981228}{[1]}
%\href{https://www.sciencedirect.com/science/article/abs/pii/S0921889021002323}{[2]}
%\href{https://www.tandfonline.com/doi/full/10.1080/00207179.2018.1514129}{[3]} --- \href{https://scholar.google.com/citations?view\_op=list\_works\&hl=en\&user=9\_hI4hMAAAAJ}{google scholar}\\

%Indicative software packages: &
%\href{https://github.com/li9i/cbgl}{\texttt{cbgl}} $\cdot$
%\href{https://github.com/li9i/fsm-lo}{\texttt{fsm-lo}} $\cdot$
%\href{https://github.com/li9i/lama-odom}{\texttt{lama-odom}} $\cdot$
%\href{https://github.com/li9i/pandora\_vision\_2014/tree/hydro-devel/pandora\_vision\_hole\_detector}{\texttt{pvhd}} ---  \href{https://github.com/li9i}{\texttt{github}} \\

%Demos / videos: &
%\href{https://www.youtube.com/watch?v=xaDKjI0WkDc}{\texttt{cbgl}} $\cdot$ \href{https://cultureid.web.auth.gr/?page\_id=200&lang=en}{\texttt{cultureid}} $\cdot$ \href{https://www.youtube.com/watch?v=hB4qsHCEXGI}{\texttt{fsm}} $\cdot$ \href{https://relief.web.auth.gr/}{\texttt{relief}} $\cdot$ \href{https://www.youtube.com/watch?v=937OZez1iN8}{\texttt{mpc}}\\


  %\href{https://raw.githubusercontent.com/li9i/portfolio/master/portfolio.pdf}{\texttt{Portfolio}} &
%\end{tabular}
%\\

%----------------------------------------------------------------------------------------
%	References
%----------------------------------------------------------------------------------------
\section{References}
\noindent Dr. Antonis Dimitriou $\cdot$ Coordinator of the projects for which I worked during my AUTh years \\
\faPhone \ +30 6978896350 $\cdot$ \faEnvelopeO \ \href{mailto:antodimi@auth.gr}{antodimi@auth.gr} \\

\noindent For a complete list of references, whether they be supervisors or coworkers, visit
\begin{center}\noindent
\href{https://github.com/li9i/cv/tree/master/references}{\texttt{https://github.com/li9i/cv/tree/master/references}}
\end{center}

\vspace*{\fill}

\end{document}

\documentclass[a4paper,10pt,twoside]{article}
\usepackage{alefilot}
\usepackage{fontspec}
\usepackage[utf8]{inputenc}
\usepackage[top=10mm, bottom=1mm, inner=20mm, outer=20mm, footskip=20mm]{geometry}
\setmainfont{Latin Modern Roman} % Main document font

% To reduce the height of the top margin uncomment:
%\addtolength{\voffset}{-1.75cm}

\usepackage{hyperref}
\definecolor{linkcolour}{rgb}{0,0.2,0.6} % Link color
\hypersetup{colorlinks,breaklinks,urlcolor=linkcolour,linkcolor=linkcolour} % Set link colors throughout the document

\usepackage{titlesec}
\titleformat{\section}{\Large\scshape\raggedright}{}{0em}{}[\titlerule]
\titlespacing{\section}{0pt}{3pt}{3pt} % Spacing around sections

% FOOTER
\usepackage{fancyhdr}
\setlength{\headheight}{15.2pt}
\setlength{\textheight}{710pt}
\setlength{\footskip}{80pt} % Distance from bottom of text to footer
\pagestyle{fancy}
\fancyhead[]{}
\renewcommand{\headrulewidth}{0pt}
\fancyfoot[LO,RE]{\hfill \small source: \href{https://github.com/li9i/cv}{\texttt{https://github.com/li9i/cv}} --- APR 2025}

% If small portions of text are spilling onto the second page, consider a manual adjustment:
\enlargethispage{2\baselineskip} % Adds two lines of vertical space

\usepackage{wasysym}
\usepackage{fontawesome}
\usepackage{tikz}

\begin{document}

\pagenumbering{gobble}

%----------------------------------------------------------------------------------------
%	NAME AND CONTACT INFORMATION
%----------------------------------------------------------------------------------------
\par{\centering{\Huge Alexandros Filotheou}\bigskip\par}

\begin{center}
{\footnotesize Thessaloniki, Greece $|$ \href{mailto:alexandros.filotheou@gmail.com}{alexandros.filotheou@gmail.com} $|$ (+30) 693 8787 677 $|$ \href{https://www.linkedin.com/in/alexandros-filotheou-5b6a8676/}{linkedin.com/in/alexandros-filotheou} \\ \href{https://github.com/li9i/}{github.com/li9i} $|$  \href{https://scholar.google.com/citations?hl=en&user=9_hI4hMAAAAJ&view_op=list_works}{scholar.google.com/alexandros-filotheou}}\vspace{1em}
\end{center}

%----------------------------------------------------------------------------------------
%	SUMMARY
%----------------------------------------------------------------------------------------
\vspace{-0.5cm}
\begin{bw_box} \small
  \hspace{1em}I have 7+ years of hands-on experience in robotics which includes SLAM, \href{https://github.com/li9i/fsm-lo}{Localisation},
  \href{https://link.springer.com/article/10.1007/s10846-019-01086-y}{Autonomous Navigation}, \href{https://www.tandfonline.com/doi/full/10.1080/00207179.2018.1514129}{Control},
  \href{https://github.com/li9i/pandora\_vision\_2014}{Computer Vision}, and
  general integration, problem-solving, and troubleshooting. These I have acquired in \href{https://relief.web.auth.gr/\%CF\%81\%CE\%BF\%CE\%BC\%CF\%80\%CE\%BF\%CF\%84\%CE\%B9\%CE\%BA\%CE\%AC-\%CE\%BF\%CF\%87\%CE\%AE\%CE\%BC\%CE\%B1\%CF\%84\%CE\%B1/}{real
  conditions with real robots} as well as in simulation, working for research
  projects funded by the European Commission and the Greek State, or through
  volunteering. I am proficient in ROS and ROS 2, in versioning through git, virtualisation via Docker, and also in MATLAB/Octave, all
  the while under Linux. My primary coding language is \href{https://github.com/li9i/fsm}{C++} and my secondary
  \href{https://github.com/cultureid-auth-ros-packages/cultureid-waypoints-following}{Python}.
\end{bw_box}


%----------------------------------------------------------------------------------------
%	LINKS
%----------------------------------------------------------------------------------------
\vspace{+0.1cm}
\begin{tabular}{rl}

  Demos / videos & \href{https://www.youtube.com/watch?v=xaDKjI0WkDc}{\texttt{cbgl}} $\cdot$ \href{https://www.youtube.com/watch?v=hB4qsHCEXGI}{\texttt{fsm}} $\cdot$ \href{https://cultureid.web.auth.gr/?page\_id=200&lang=en}{\texttt{cultureid}} $\cdot$ \href{https://relief.web.auth.gr/}{\texttt{relief}} $\cdot$ \href{https://www.youtube.com/watch?v=937OZez1iN8}{\texttt{multi-mpc}} $\cdot$ \href{https://docs.google.com/viewer?url=https://raw.githubusercontent.com/li9i/portfolio/master/portfolio.pdf}{Portfolio} \\

  Indicative software packages &
  \href{https://github.com/li9i/cbgl}{\texttt{cbgl}} $\cdot$
  \href{https://github.com/li9i/fsm-lo}{\texttt{fsm-lo}} $\cdot$
  %\href{https://github.com/li9i/lama-odom}{\texttt{lama-odom}} $\cdot$
  \href{https://github.com/li9i/pandora\_vision\_2014/tree/hydro-devel/pandora\_vision\_hole\_detector}{\texttt{PANDORA Hole Detection}} \\

  Indicative publications &
  \href{https://ieeexplore.ieee.org/abstract/document/10802235}{[\texttt{cbgl}]} $\cdot$
  \href{https://ieeexplore.ieee.org/abstract/document/9981228}{[\texttt{fsm}]} $\cdot$
  \href{https://www.tandfonline.com/doi/full/10.1080/00207179.2018.1514129}{[\texttt{multi-mpc}]} \\

\end{tabular}

%----------------------------------------------------------------------------------------
%	WORK EXPERIENCE
%----------------------------------------------------------------------------------------
\section{Experience}

  \noindent\textbf{Robotics Engineer} $\cdot$ CERTH, Thessaloniki GR \hfill Sep 2023 -- Present \\
  \begin{minipage}[t]{\textwidth}
    \begin{itemize}
      \item Head of software integration and sole git repository maintainer of EU-funded R\&D project \href{https://www.robetarme-project.eu/}{RoBétArmé}\vspace{-0.8em}
      \item Organised and deployed ROS and ROS 2 nodes and other software in over 50 Docker images over real \& simulated ground mobile bases and supporting computing units\vspace{-0.8em}
      \item Bridged ROS with ROS 2 in terms of interfaces and over multiple machines
    \end{itemize}
  \end{minipage} \\[0.4em]

  \noindent\textbf{Robotics and Control Engineer} $\cdot$ ECE dept., Aristotle University of Thessaloniki GR \hfill Sep 2018 -- Mar 2023 \\
  \begin{minipage}[t]{\textwidth}
    \begin{itemize}
      \item Technical lead of everything robotics in NSRF R\&D projects \href{https://relief.web.auth.gr/language/en/home/}{Relief} and \href{https://cultureid.web.auth.gr/?page\_id=200&lang=en}{CultureId} \vspace{-0.8em}
      \item Implemented and deployed SLAM, autonomous exploration \& navigation, 3D reconstruction, and corresponding friendly GUIs for real \& simulated ground and aerial mobile bases\vspace{-0.8em}
      \item Decreased RFID-tag localisation error by 4x by decreasing and robustifying LIDAR-derived particle and Kalman filter robot pose error by 7x \vspace{-0.8em}
      \item Implemented and deployed mobile robot-human gui-interactive applications which have served more than 3000 visitors at the Archaeological Museum of Thessaloniki from early '23 to date \vspace{-0.8em}
      \item Coauthored \& supported implementation behind eighteen high-ranking journal \& conference publications
    \end{itemize}
  \end{minipage} \\[0.4em]

  \noindent\textbf{Teaching Assistant} $\cdot$ KTH Royal Institute of Technology, Stockholm SE \hfill Sep 2016 -- Nov 2016 \\
  \begin{minipage}[t]{\textwidth}
    \begin{itemize}
      \item Examiner for \textit{DD2380 - Artificial Intelligence}, supervised by Prof. Patric Jensfelt
    \end{itemize}
  \end{minipage}

%----------------------------------------------------------------------------------------
%	VOLUNTARY EXPERIENCE
%----------------------------------------------------------------------------------------
\section{Voluntary Experience}

\textbf{Computer Vision Engineer} $\cdot$ \href{https://issel.ee.auth.gr/pandora-robotics/}{PANDORA Robotics}, Thessaloniki GR \hfill Oct $2013$ -- Jul $2014$
  \begin{minipage}[t]{\textwidth}
    \begin{itemize}
      \item Implemented and deployed detection of holes on walls through Microsoft's Kinect RGBD camera in \texttt{C++} in the context of the international competition RoboCup Rescue
    \end{itemize}
  \end{minipage}

%----------------------------------------------------------------------------------------
%	EDUCATION
%----------------------------------------------------------------------------------------
\section{Education}

\noindent \href{https://ikee.lib.auth.gr/record/354644}{\textbf{Doctorate}} $\cdot$ Electrical \& Computer Engineering $\cdot$ Aristotle University of Thessaloniki \hfill Sep $2018$--Jun $2023$ \\
\href{http://kth.diva-portal.org/smash/record.jsf?pid=diva2\%3A1102597\&dswid=2875}{\textbf{Master of Science}} $\cdot$ Systems, Control, and Robotics $\cdot$ KTH Royal Institute of Technology \hfill Sep $2015$--Jun $2017$ \\
\href{https://ikee.lib.auth.gr/record/291560}{\textbf{Diploma}} $\cdot$ Electrical \& Computer Engineering $\cdot$ Aristotle University of Thessaloniki \hfill Sep 2005--Jul $2013$

%----------------------------------------------------------------------------------------
% SKILLS
%----------------------------------------------------------------------------------------
\section{Skills}

\begin{tabular}{rp{12cm}}
Languages & \texttt{C/C++}, \texttt{Python}, \texttt{shell}, \texttt{MATLAB/Octave} \\
$\{$Meta-$\}$Operating Systems & \texttt{Linux}, \texttt{ROS 2}, \texttt{ROS} \\
  Tools & \texttt{git}, \texttt{Docker}, Eigen, Behavior Trees, Gazebo, \texttt{CI}/\texttt{CD}, \texttt{Qt} / \texttt{Tkinter}, \texttt{OpenCV} \\
Control Techniques & MPC, PID, LQR
\end{tabular}

%----------------------------------------------------------------------------------------
%	LANGUAGES
%----------------------------------------------------------------------------------------
\section{Languages}

\begin{tabular}{rp{12cm}}
  English & Fluent --- IELTS Score $8.5$ \ \ (Greek Native)
\end{tabular}

\vspace{0.2cm}

%----------------------------------------------------------------------------------------
%	References
%----------------------------------------------------------------------------------------
\section{References}
\noindent Dr. Antonis Dimitriou $\cdot$ Coordinator of R\&D projects $\cdot$ (+30) 697 88 96 350 $\cdot$ \href{mailto:antodimi@auth.gr}{antodimi@auth.gr} \\

\noindent For a complete list of references, whether they be supervisors or coworkers, visit
\begin{center}\noindent
\href{https://github.com/li9i/cv/tree/master/references}{\texttt{https://github.com/li9i/cv/tree/master/references}}
\end{center}

\end{document}

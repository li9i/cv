\documentclass[a4paper,10pt,twoside]{article}
%\usepackage[top=1mm, bottom=1mm, inner=20mm, outer=20mm, footskip = 1mm]{geometry}
\usepackage{fullpage}
\usepackage{fontspec}
\usepackage[utf8]{inputenc}

%\defaultfontfeatures{Mapping=tex-text}
%\setmainfont[Mapping=tex-text]{GFS Didot} % Main document font
%\setmainfont{cmr12} % Main document font
\setmainfont{GFSdidot} % Main document font

\usepackage{wasysym}
\usepackage{fontawesome}

\usepackage[usenames,dvipsnames]{xcolor} % Required for specifying custom colors

%\usepackage[big]{layaureo} % Margin formatting of the A4 page, an alternative to layaureo can be
\usepackage{fullpage}
% To reduce the height of the top margin uncomment:
\addtolength{\voffset}{-1.3cm}

\usepackage{hyperref} % Required for adding links	and customizing them
\definecolor{linkcolour}{rgb}{0,0.2,0.6} % Link color
\hypersetup{colorlinks,breaklinks,urlcolor=linkcolour,linkcolor=linkcolour} % Set link colors throughout the document

\usepackage{titlesec} % Used to customize the \section command
\titleformat{\section}{\Large\scshape\raggedright}{}{0em}{}[\titlerule] % Text formatting of sections
\titlespacing{\section}{0pt}{3pt}{3pt} % Spacing around sections


\usepackage{fancyhdr}
\setlength{\headheight}{15.2pt}
\setlength{\textheight}{710pt}
\pagestyle{fancy}
\fancyhead[]{}
\renewcommand{\headrulewidth}{0pt}
\fancyfoot[LO,RE]{\small Θεσσαλονίκη, Ιούλιος 2024 \hfill επικοινωνία: 693 87 87 677 $|$ alexandros.filotheou@gmail.com\\
\hfill πηγή: \href{https://github.com/li9i/CV}{https://github.com/li9i/CV}}

\usepackage{tikz}

\begin{document}

%\pagestyle{plain} % Removes page numbering
\pagenumbering{gobble}

%----------------------------------------------------------------------------------------
%	NAME AND CONTACT INFORMATION
%----------------------------------------------------------------------------------------

\par{\centering{\Huge Αλέξανδρος Φιλοθέου}\bigskip\par}

%----------------------------------------------------------------------------------------
%	WORK EXPERIENCE
%----------------------------------------------------------------------------------------

\section{Εργασιακή Εμπειρία}

\begin{tabular}{rp{12cm}}
09.2023  $-$ \hfill παρόν \hfill  & \textbf{Μεταδιδακτορικός Ερευνητής} \\
                    & Εθνικό Κέντρο Έρευνας \& Τεχνολογικής Ανάπτυξης, Θεσσαλονίκη \\
&\\
09.2018 $-$ 03.2023 & \textbf{Εργολήπτης Ερευνητικών Έργων Ρομποτικής} \\
                    & Τμήμα Ηλεκτρολόγων Μηχανικών \& Μηχανικών Υπολογιστών, Α.Π.Θ.\\
&\\
09.2016 $-$ 11.2016 & \textbf{Teaching Assistant} $\cdot$ DD2380 Artificial Intelligence \\ & KTH Royal Institute of Technology, Stockholm, Sweden\\
&\\
%------------------------------------------------
10.2011 $-$ 03.2012 & \textbf{Σχεδιαστής Βάσεων Δεδομένων} \\
%------------------------------------------------
03.2011 $-$ 05.2011 & \textbf{Προγραμματιστής Βάσεων Δεδομένων} $\cdot$ Πρακτική Άσκηση \\ & Εγνατία Οδός Α.Ε., Θεσσαλονίκη \\
&\\
%------------------------------------------------
07.2008 $-$ 06.2009 & \textbf{Μηχανικός Τηλεπικοινωνιών} $\cdot$ Ο.Τ.Ε., Θεσσαλονίκη\\
\end{tabular} \\


%----------------------------------------------------------------------------------------
%	VOLUNTARY EXPERIENCE
%----------------------------------------------------------------------------------------

\section{Εθελοντική Εμπειρία}

\begin{tabular}{rp{12cm}}
10.2013 $-$ 07.2014 & \textbf{Μηχανικός Υπολογιστικής Όρασης} $\cdot$ Ομάδα ρομποτικής PANDORA, Τμήμα Ηλεκτρολόγων Μηχανικών \& Μηχανικών Υπολογιστών, Α.Π.Θ.\\
\end{tabular} \\


%----------------------------------------------------------------------------------------
%	EDUCATION
%----------------------------------------------------------------------------------------

\section{Εκπαίδευση}

\begin{tabular}{rp{11cm}}
09.2018 $-$ 06.2023 & \textbf{Διδακτορικό Δίπλωμα} \\
                    & Τμήμα Ηλεκτρολόγων Μηχανικών \& Μηχανικών Υπολογιστών, Α.Π.Θ.\\
&\\
09.2015 $-$ 06.2017 & \textbf{Μεταπτυχιακό Δίπλωμα} $\cdot$ Systems, Control, and Robotics\\
                    & School of Electrical Engineering and Computer Science \\
                    & KTH Royal Institute of Technology, Stockholm, Sweden\\
&\\
09.2005 $-$ 07.2013 & \textbf{Δίπλωμα} \\
                    & Τμήμα Ηλεκτρολόγων Μηχανικών \& Μηχανικών Υπολογιστών, Α.Π.Θ.\\
                    & {\small Βαθμός: 7.94 / 10 $\cdot$ Κατάταξη: 23 / 280}
\end{tabular}\\


%----------------------------------------------------------------------------------------
%	COMPUTER SKILLS
%----------------------------------------------------------------------------------------
\section{Γνώσεις Υπολογιστών}

\begin{tabular}{rp{9cm}}
Γλώσσες                             & \texttt{C/C++, Python, shell, Octave/MATLAB, LaTeX} \\
$\{$Μετα-$\}$λειτουργικά Συστήματα  & \texttt{Linux, ROS 2, ROS} \\
  Εργαλεία / Βιβλιοθήκες            & \texttt{git, OpenCV, Docker}, \texttt{Qt (cpp) | Tkinter (py)}
\end{tabular} \\


%----------------------------------------------------------------------------------------
%	LANGUAGES
%----------------------------------------------------------------------------------------

\section{Γλώσσες}
\begin{tabular}{rp{12cm}}
Αγγλικά & C2 (IELTS score 8.5) \\
Ελληνικά & Μητρική
\end{tabular} \\


%----------------------------------------------------------------------------------------
%	LINKS
%----------------------------------------------------------------------------------------
\section{Σύνδεσμοι}

\begin{tabular}{rp{12cm}}
  Ενδεικτικές δημοσιεύσεις: &
\href{https://ieeexplore.ieee.org/abstract/document/9981228}{[1]}
\href{https://www.sciencedirect.com/science/article/abs/pii/S0921889021002323}{[2]}
\href{https://www.tandfonline.com/doi/full/10.1080/00207179.2018.1514129}{[3]} --- \href{https://scholar.google.com/citations?view\_op=list\_works\&hl=en\&user=9\_hI4hMAAAAJ}{google scholar}\\

  Ενδεικτικά πακέτα λογισμικού&
\href{https://github.com/li9i/cbgl}{\texttt{cbgl}} $\cdot$
\href{https://github.com/li9i/fsm-lo}{\texttt{fsm-lo}} $\cdot$
\href{https://github.com/li9i/lama-odom}{\texttt{lama-odom}} $\cdot$
\href{https://github.com/li9i/pandora\_vision\_2014/tree/hydro-devel/pandora\_vision\_hole\_detector}{\texttt{pvhd}} --- \href{https://github.com/li9i}{\texttt{github}}\\

  Demos / videos: & \href{https://www.youtube.com/watch?v=xaDKjI0WkDc}{\texttt{cbgl}} $\cdot$ \href{https://www.youtube.com/watch?v=hB4qsHCEXGI}{\texttt{fsm}} $\cdot$ \href{https://cultureid.web.auth.gr/?page\_id=200&lang=en}{\texttt{cultureid}} $\cdot$ \href{https://relief.web.auth.gr/}{\texttt{relief}} $\cdot$ \href{https://www.youtube.com/watch?v=937OZez1iN8}{\texttt{mpc}}\\

  \href{https://raw.githubusercontent.com/li9i/portfolio/master/portfolio.pdf}{\texttt{Portfolio}} &
\end{tabular}
\\

%----------------------------------------------------------------------------------------
%	References
%----------------------------------------------------------------------------------------
\section{Συστάσεις}
\noindent Δρ. Αντώνης Γ. Δημητρίου: Συντονιστής των ερευνητικών έργων στα οποία εργάσθηκα στο Α.Π.Θ. \hspace{1cm}
\faPhone \ 6978896350 $\cdot$ \faEnvelopeO \ \href{mailto:antodimi@auth.gr}{antodimi@auth.gr} \\



%----------------------------------------------------------------------------------------
% Contact
%----------------------------------------------------------------------------------------
%\section{Επικοινωνία}

%\begin{center}
  %\faPhone \ 693 87 87 677 \ $\cdot$ \ \faEnvelopeO \ \href{mailto:alexandros.filotheou@gmail.com}{alexandros.filotheou@gmail.com}
%\end{center}

\end{document}
